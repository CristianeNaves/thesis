Sistemas Multi-Robôs (SMR) são sistemas que consistem em mais de um agente robótico. Por algumas décadas esses sistemas foram utilizados em diversos contextos para cumprir diversas tarefas, especialmente em ambientes dinâmicos. SMRs atuam em um espaço ciber físico (i.e. parte do “mundo real”), logo seus agentes estão propensos à mudanças provenientes tanto de outros agentes do sistema como do ambiente em que estão inseridos \cite{iocchi2000reactivity}. Para aumentar a adaptabilidade do SMR, pode-se projetá-lo como um sistema auto-adaptativo, tornando-o capaz de responder à mudanças no ambiente, de maneira a continuar cumprindo objetivos e respeitando os limites impostos ao sistema \cite{sykes2010autonomous}.

Os agentes desses sistemas (i.e. robôs) existem no mundo físico e interagem com ele e entre si de maneiras mais complexas do que agentes de outros sistemas (e.g. computadores, bancos de dados, etc) \cite{cao1997cooperative}. Isso traz desafios para o desenvolvimento desse tipo de sistema, principalmente a preparação de experimentos com vários robôs \cite{noori20173d}. Essa dificuldade pode ser superada com o uso de simuladores.

Simuladores podem ser empregados tanto para testar a segurança, eficiência e robustez do sistema, quanto para prototipação de SMRs e robôs \cite{noori20173d}, \cite{pinciroli2012argos}. Outras vantagens de simuladores incluem: 

\begin{itemize}
    \item Menor custo de tempo e recursos para preparação e execução do experimento
    \item Ambientes simulados podem ser mais ricos, complexos e seguros que ambientes reais ou em laboratório \cite{robotSimulation}
    \item é possível testar hardware que não está disponível \cite{pinciroli2012argos},  \cite{echeverria2011morse}, \cite{robotSimulation}
\end{itemize}

De acordo com o Survey realizado \cite{robotSimulation}, muitos desenvolvedores de sistemas robóticos confirmam que simulações são ferramentas populares entre eles e um dos casos de uso mais comuns são os testes.

Diversos simuladores para SMRs existem na literatura, por exemplo Gazebo \cite{koenig2004gazebo}, Simbad \cite{hugues2006simbad}, CoppeliaSim \cite{rohmer2013coopeliasim}, MORSE \cite{echeverria2011morse} e Dragonfly \cite{maia2019dragonfly}, entre outros. Cada um desses simuladores foi criado com propostas diferentes, desde simulação precisa das partes que compõem um robô e sua interação com o ambiente (Gazebo, CoppeliaSim, Morse), até simulações de mais alto nível focando principalmente no comportamento dos robôs (Simbad, Dragonfly). Simulações multi-robô são suportadas por simuladores atuais, mas geralmente em menor número - devido ao alto uso de recursos computacionais necessários para simular cada robô (i. e. experimentos feitos com Gazebo mostraram que o simulador tem dificuldades ao simular mais de 10 robôs \cite{noori20173d}) - ou são muito específicos quando conseguem simular mais robôs (i.e. Dragonfly supostamente é capaz de simular até 400 entidades, mas está restrito à simulação de drones \cite{maia2019dragonfly}).

Ainda que simulações tenham um bom potencial, testes de campo são mais utilizados para validação e para verificação em sistemas robóticos do que os testes baseados em simulações. Alguns problemas dificultam o uso de simulações para fazer a validação e verificação (V\&V), dentre eles \cite{robotSimulation}:
\begin{itemize}
    \item Não ser possível ou dar muito trabalho para testar a aplicação com a GUI desabilitada;
    \item Não ser possível ou dar muito trabalho para executar a simulação sem intervenção manual;
    \item Não ter uma terminação clara da simulação (fechar a simulação no final da execução e mostrar os erros caso tenha algum);
    \item Interfaces instáveis;
    \item Dificuldades ao tentar reproduzir um resultado.
\end{itemize}

Mesmo com os diversos desafios, o autor \cite{robotSimulation} conclui que os desenvolvedores estão usando simulações para testar os seus robôs e muitos querem incorporar a simulação na rotina de automação de testes, mas é também sugerido fazer algumas mudanças:
\begin{itemize}
    \item Prover suporte para simulações sem a parte gráfica (GUI);
    \item Suporte para APIS estáveis e que tenham um bom design;
    \item Ter uma forma de reproduzir os resultados;
    \item Reduzir os custos de hardware (no caso de simulações pesadas).
\end{itemize}

O Laboratório de Engenharia de Software (LES) da Universidade de Brasília (UnB) conduz pesquisas na área de sistemas multi-agentes, incluindo sistemas multi-robôs. Entre os simuladores empregados nas pesquisas do LES, encontram-se Gazebo e MORSE, porém tem sido relatadas dificuldades com o uso desses simuladores em cenários com times maiores de robôs. Isso se dá pelo alto nível de detalhamento físico das simulações, que exige recursos computacionais consideráveis. Quando o objetivo da pesquisa é mais voltado para os algoritmos que coordenam os diferentes agentes do sistema, esse nível alto de detalhamento é desnecessário, mas aumenta consideravelmente o tempo de cada experimento.

Nesse cenário, parte da proposta deste projeto é fornecer uma ferramenta direcionada para simulação de sistemas multi-robôs auto-adaptativo com baixo nível de detalhamento físico. Esta ferramenta será usada na avaliação e comparar algoritmos de distribuição de tarefas entre agentes de um SMR. Também pode ser usado para prototipação das características de cada agente do SMR, bem como validação de requisitos do time de robôs de algum sistema.

Outra proposta deste projeto é como testar de forma automatizada as missões de uma simulação robótica, levando em consideração as dificuldades citadas anteriormente. Com este fim, será utilizado o Behavior Driven Development (BDD) para fazer a verificação e a validação dos comportamentos das missões.

\gio{Acho que a descrição do BDD tinha que ficar só no caítulo 2. Apresentar ele, como no parágrafo de cima parece o bastante. E eu diria também usar verbos no futuro (e.g. será usado) poderia ser trocada pro passado mesmo, porque a gente já fez o trabalho.}

Uma das ideias do BDD é validar o comportamento de uma determinada parte do software utilizando uma linguagem que pode ser entendida por todas as partes interessadas \cite{bddInAction}. O BDD será usado para criar diversos cenários de missões de simulação e verificar o comportamento da execução destes cenários.  Os cenários são compostos de várias regras e cada regra pode ser mapeada a um método que vai verificar se ela é executada corretamente. Para a criação dos cenários é utilizada linguagens que podem ser entendidas por todos os stakeholders, um exemplo é a linguagem Gherkin.


\gio{Melhor passar essa descrição de ECS pro capítulo 2, pra não ficar repetido}

Nesse padrão existem três elementos principais: as entidades, os componentes e os sistemas \cite{ecs_mmog_development}. As entidades representam um objeto único da simulação e é composta na maioria das vezes por apenas um identificador. Os outros dados relacionados à Entidade ficam guardados em outra estrutura, conhecida como Componente. 

O Componente é um container de dados que representa um aspecto do objeto, por exemplo, um componente de posição ou de renderização. Os Componentes possuem apenas dados, eles não possuem nenhuma lógica, nenhuma função para processar esses dados. Os responsáveis pela parte lógica são os Sistemas. 

Cada sistema será responsável por prover uma lógica, por exemplo, um Sistema de Renderização vai mostrar uma imagem na tela e um Sistema de Movimentação vai mover um objeto de um lugar para outro. Para realizar todos os processamentos, cada sistema receberá todos os componentes necessários para o seu funcionamento.  

Além do objetivo principal deste projeto, que é verificar a compatibilidade do BDD para testar de forma automatizada missões em um ambiente de robótica controlado, também será verificado se a arquitetura proposta facilitou o desenvolvimento dos testes e a criação dos cenários ou se trouxe alguma dificuldade adicional.

\section{Objetivos}

\gab{reformular os objetivos de forma a englobar outros aspectos do simulador que não só BDD}

O objetivo principal deste trabalho é analisar a compatibilidade do BDD com verificações de missões em ambiente de robótica.
\subsubsection{Objetivos Específicos}


Para analisar a compatibilidade do BDD com verificações de missões em ambiente de robótica, foram estabelecidos objetivos específicos:

\begin{itemize}
    \item Implementação da infraestrutura de simulação para o cenários (mapas, robôs, ações);
    \item Construir uma API para auxiliar nos testes das especificações das missões;
    \item Avaliar o desempenho do uso do BDD em ambiente de simulação de robótica.
\end{itemize}